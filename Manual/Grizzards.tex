%
% Grizzards printable manual
%
% This needs a lot of work yet.
%
% If you're a human, skip to "\mainmatter" to skip over the preamble.
%
\documentclass[10pt,twoside,openright]{memoir}
\usepackage{geometry}
\geometry{height=6in,width=4in}
\usepackage[utf8]{inputenc}
\usepackage{babel}
\usepackage{tgtermes}
\usepackage{hyperref}
\usepackage[protrusion=true,expansion=true]{microtype}

\checkandfixthelayout

\title{Grizzards Player's Guide}
\author{Bruce-Robert Pocock}

%% TV Standard

\ifdefined\TVNTSC
  \newcommand\TV{NTSC}
  \newcommand\REGION{US, Canada, Mexico, Japan}
\else
  \ifdefined\TVPAL
    \newcommand\TV{PAL}
    \newcommand\REGION{Europe (except France)}
  \else
    \ifdefined\TVSECAM
      \newcommand\TV{SECAM}
      \newcommand\REGION{France, Russia, Africa}
    \else
      \error{Must define a TV standard}
    \fi
  \fi
\fi

%% BEGIN TITLE

\makeatletter
\def\maketitle{%
  \null
  \thispagestyle{empty}%
  \vfill
  \begin{center}\leavevmode
    \normalfont
    {\LARGE\raggedleft \@author\par}%
    \hrulefill\par
    {\huge\raggedright \@title\par}%
    \vskip 1cm
  \end{center}%
  \vfill
  \null
  \cleardoublepage
  }
\makeatother

%%% BEGIN DOCUMENT

\begin{document}

\let\cleardoublepage\clearpage


\maketitle

\frontmatter

\null\vfill

\begin{flushleft}
\textit{Grizzards}


Copyright © 2020, Bruce-Robert Pocock

This version is for systems in \REGION{} using the \TV{} television
standard. For Atari 2600 or 7800 with AtariVox.




\thedate

\bigskip

All rights reserved.

\end{flushleft}
\let\cleardoublepage\clearpage

\mainmatter

\chapter{Introduction}

Long ago,  the world was  troubled by  monsters. The Three  Sages united
their powers at the temples to protect the villages of Aclypt.

The Sage Uron in the Temple of  the Sea protected the rivers, lakes, and
beaches  from  sea monsters.  The  Sage  Eril  in  the Temple  of  Lands
protected from  monsters on the ground.  The Sage Ybor in  the Temple of
the Sky protected from flying monsters.

For many  years this magic has  protected Aclypt. Your village  of Hapri
has been safe and quiet your entire life.

Recently, though, a traveler named  Ferris reached your village from the
nearby town  of Albron. He  tells you that  Albron is being  attacked by
monsters from the nearby forest  of Prulin. Your village's elder, Roton,
has chosen you to go to Albron and help protect them.

Your mission  is to  defend Albron,  and then  continue onward  into the
greater  world to  the Temple  of Lands  to see  why its  power did  not
protect Albron.

\chapter{Setting Up}

To play Grizzards, you will need:

\begin{itemize}
\item An Atari 2600 Video Computer System or Atari 7800 ProSystem
\item A TV or video display
\item A joystick controller
\item An AtariVox device with (optional) speakers or headphones
\item The Grizzards game cartridge
\end{itemize}

Set up your  Atari Video Computer System or ProSystem  with your TV or
video  display.  Connect  a  joystick controller  to  the  \emph{left}
controller  port,   and  the  AtariVox  device   to  the  \emph{right}
controller port.  Make certain  that the  Power switch  is in  the Off
position.

Finally, insert the Grizzards game cartridge with the
label facing up into the cartridge slot, and turn the Power On.

\section{Controller Ports}

Your joystick  connects to the  left controller port, and  your AtariVox
device connects to the right controller port.

If your AtariVox is missing, you'll see a red Sad Face screen.

Always   turn  Off   the  Power   switch  before   adding  or   removing
your AtariVox.

This  game  saves  your  progress   to  the  AtariVox  while  you  play.
Always return to the Title Screen  by pressing Game Reset before turning
Off the Power to be sure your game record is not damaged.

\section{Console Controls}

\subsection{Color/B\&W Switch (Pause)}

The Color/B\&W switch on the Atari  2600 Video Computer System, or the
Pause button  on the Atari 7800  ProSystem, can be used  to pause your
game. Push  the control into the  Color position to play,  or the B\&W
position to pause the game.

\subsection{Game Select}

When viewing the Title Screen, you can use the Game Select switch to
choose the Slot you wish to use.  See Getting Started on page
\ref{Getting Started} for details.

\subsection{Game Reset}

When viewing  the Select  Slot screen,  press the  Game Reset  switch to
begin playing the  game. While you are playing the  game, press the Game
Reset switch to save your place and return to the Title Screen.

\subsection{Difficulty Switches}

In order to erase your game progress information, both Difficulty
Switches must be moved to the A (Expert) position.

To protect  against accidentally erasing  your game record, you  can set
the right Difficulty Switch to the B (Novice) position.

While  you  are playing,  the  Difficulty  Switches  have no  effect  on
game play. 



Grizzards

The land of Syrex is a dangerous place. Fierce monsters roam the countryside.
But luckily for you, you're a Grizzard handler!

Train your Grizzard to use a variety of moves to take on the monsters.
Discover new kinds of Grizzards with new capabilities. Can you conquer all
the monsters of Syrex?

In the Grizzards videogame, you'll roam the land looking for monsters.
Monsters may surprise you as you travel, or you may see them coming. When
faced with terrifying beasts, you'll direct your Grizzard to use its moves
to defend you or attack the monsters.



How To Begin

Once your Atari system is set up and everything is connected, turn on the
Power switch. You'll see a brief introduction and title screen appear. If
you have an AtariVox device, you'll also hear the title spoken.

Press the Game Select switch to choose a memory slot for your game. There
are 3 memory slots possible. Press Game Select to rotate through them. If
someone has already begun to play Grizzards in a certain slot, your screen
will show ``RESUME.'' If a slot is empty, you'll see ``BEGIN'' instead.

If you want to destroy (forever!) a game in progress, *** press the joystick
to the left, then to the right, and press the Fire button. Your screen will
change to show that you can now erase a memory slot. 

When you have selected the slot you want to begin (or resume), press the
Game Reset switch to start.

Choosing a Starter Grizzard

When you begin the game, you can roam the game world. You'll see that there
are four directions in which you can set out. In each direction is a
Grizzard companion that you can choose to join your party. You can only have
one Grizzard companion, so choose carefully. 

*** explain starter Grizzards a bit.


How To Play

General Controls

You'll use the joystick controller in the left controller port for most of
your game play needs. Depending on where you are in the game, the joystick
will be used in different ways, which will be explained next.

The Game Select switch is used from the title screen to choose a slot in
which to save your progress on your MemCard or AtariVox. When you have
selected a slot, the Game Reset switch will start the game.

While you're playing the game, the Game Select switch can be used to see the
statistics of your current Grizzard companion, and the Game Reset switch can
be used to leave the game and return to the title screen, saving your
current progress to your MemCard or AtariVox.

The Color/B&W switch (on an Atari 7800: Pause button) can be used to pause
game play. The screen will show ``PAUSED'' and you will not be able to
continue playing until you toggle the switch to the Color position.

This game does not make use of the Difficulty Switches.


Roaming The World

The World Map screen shows your current score (initially 000000) at the top
of the screen.  Below this, you'll see the current area in which you are
traveling.  Guide yourself using the joystick controller.

*** image

Throughout the world you may encounter monsters. Sometimes these monsters
may sneak up on you and attack! Other times you'll see them waiting for you
and can avoid them --- or walk up to them when you're ready to face them.

You may also encounter Grizzards in the world that you can convince to join
your party. When you first begin, you'll only be able to choose one
Grizzard. Later, you'll have the option to swap Grizzards in your party,
with up to 3 additional Grizzards saved in your MemCard or AtariVox device.
Each Grizzard may have its own moves and skill scores. To review the
statistics of your current Grizzard companion, press Game Select.

To swap which Grizzard you're using, or to heal your current Grizzard
partner, you'll need to find a Grizzard Station like this: ***

At a Grizzard Station your Grizzard partner will be fully healed. You can
then use the joystick to choose a different Grizzard to travel with you.


Battling Monsters

Monsters plague the world of Syrex. If you're caught by monsters without a
Grizzard partner, they're sure to eat you alive! Luckily your Grizzard
partner will defend you from them, and monsters will attack it before you.

When you encounter monsters, you'll see the Combat display, which looks like
this: ***

The name of the monsters that you're facing is displayed. Monsters may
travel in groups, so you may see more than one monster facing you.

You'll also see the name of your Grizzard companion and its portrait below.
The long bar represents the health of your Grizzard. If it is reduced to
zero, your adventure will be over.

Using the joystick controller, you can choose from among the moves that your
Grizzard knows how to perform. Press up and down to select a move, or press
left to attempt to flee from the monsters. When you see the selection you
want, press the Fire button.

Most moves are targetted against a single monster.  When you use one of
these moves, if you are facing multiple monsters, use the Joystick
controller to choose which monster to target with the move and press the
Fire button.

To review the statistics of your Grizzard, press Game Select.

You will see a special display when your Grizzard, or a monster, is
executing a move: ***

The creature using that move's name will appear first, then the name of the
move, then the creature being targeted. If you have speakers or headphones
on your AtariVox, you'll also hear the move announced.

It's possible for a move to miss its target. If that happens, you'll see
MISSED appear briefly (and hear it announced on your AtariVox).

After a move has been executed, the creature targeted by that move may be
injured (lose hit points) or have its statistics changed. Changes to
statistics are temporary and last only the duration of one battle. After the
battle, your Grizzard's statistics will return to normal.

If you Grizzard loses hit points, the bar below your Grizzard will reduce in
length. When your Grizzard is nearly out of hit points, the bar will change
to a pulsating red color to draw attention to that fact.

If your Grizzard's statistics are affected by a move, you'll see the
Statistics display for your Grizzard. (You'll also hear the change announced
on your AtariVox.) Press the Fire button to continue.

If your Grizzard is defeated, the monsters will surely eat you alive. Your
adventure will end there, and you'll return to the title screen. Don't
worry, though; you can continue from your previous location by resuming your
game's saved progress. Just press Game Select to choose the same game slot
and Game Reset to resume and try again.

If you defeat all of the monsters, victory is yours! Your score will
increase, and you'll return to the World Map screen victorious.





Grizzard Stations

A Grizzard Station is a place where you can heal your Grizzard's wounds and
swap between Grizzard characters in your party to choose your current
companion.

When you encounter a Grizzard Station, you'll see this screen: ***

Your Grizzard will immediately be healed completely; you don't need to do
anything else if that's what you meant to do.

The name of your current Grizzard companion appears here, along with their
portrait.

To swap between Grizzards in your party, press the joystick up and down. If
you have additional Grizzards in your party, they'll switch places to became
your companion immediately.

You can press Game Select switch to review the statistics of
your selected companion.

To leave the Grizzard Station, press the Fire button.



Banishing A Grizzard

You can have up to four Grizzards in your party, only one of which is your
current companion.

Once you have more than one Grizzard in your party, you can also banish a
Grizzard.  Banished Grizzards will leave your party forever.  

You must enter a Grizzard Station to banish a Grizzard.

To banish a
Grizzard from your party, move the joystick to the left, then to the right. 
The screen will turn red and you will see the prompt ``BANISH'' with your
Grizzard's name. To banish the Grizzard, press the Fire button. To keep the
Grizzard in your party, press the joystick in any direction.

When you banish a Grizzard, the next Grizzard from your party will
immediately become your new companion.



Winning the Campaign

If you defeat all of the main monsters in the game, you'll see a fireworks
show to let you know that you've completed the campaign. Your ``final''
score will appear above the fireworks display.

After this reward, you'll return to the Title Screen. You can resume play
from your save game slot if you'd like, and continue to battle sneaky
monsters that attack you in some areas.



\chapter{Getting Started}\label{Getting Started}

When you first turn on Grizzards, you'll see the Title
Screen. To begin playing, press the Game Select switch.

In  Grizzards,  your game  progress is  recorded on  your
AtariVox  device. You  have  a choice  of 3  ``slots''  so that  three
different players  can use  the same AtariVox  device to  record their
progress.

Press Game Select to choose one of the three slots, then press Game
Reset.

If you've started an adventure in a slot, you'll see that the word
``RESUME'' appears on your screen. If that slot is empty, you'll
instead see the word ``BEGIN.''

\section{Resuming Your Progress}

When you leave the game, either by pressing Game Reset or reaching a
``GAME OVER,'' your progress will automatically be saved on your
AtariVox. Your progress is also saved at certain points in the story.

At first, when you resume the game, you'll return to your home village
of Hapri. Throughout the game, however, you may encounter certain
characters who will ``bless'' you. From that point on, your play will
resume at the place where you were last blessed.

\section{Starting Your Adventure Over}\label{Starting Your Adventure Over}

When you choose a slot with no game record in it already, you'll begin
a new adventure in your home village of Hapri.

If you want to delete your adventure and start again in Hapri, it's a
little bit tricky. This is to make sure you don't accidentally lose
your progress!

From the Title Screen, press Game Select. Then, press Game Select
until the slot you want to erase is shown.

Here's the tricky part. You'll need to:

\begin{itemize}
\item Make sure that both of the Difficulty Switches on your console
  are set to the ``A'' or ``Expert'' position.
\item With your joystick controller, pull down (toward you) on the
  joystick and hold down the Fire button.
\end{itemize}

The screen will change from saying ``SELECT SLOT'' to saying ``ERASE
SLOT.'' The text will also be red to catch your attention.

If  you're  \emph{sure}   you  want  to  erase  your   game  data,  then
\emph{without}  letting go  of the  Fire  button, move  the joystick  up
(toward  the TV).  You'll  see that  the  slot changes  from  IN USE  to
VACANT immediately.

\emph{Once your  game record has been  erased, there's no way  to get it
  back, so think carefully before you erase it.}

\subsection{Protecting Your Game Record}

If the Right  Difficulty Switch is in the B  (Novice) position, then you
can't  erase a  game slot.  When you  connect your  AtariVox, check  the
position of that switch.

If your AtariVox is missing, you'll see a red Sad Face screen.

Always   turn  Off   the  Power   switch  before   adding  or   removing
your AtariVox.

This game saves your progress to the AtariVox while you play.  Always
return to the Title Screen by pressing Game Reset before turning Off
the Power to be sure your game record is not damaged.


\chapter{Interacting With the Game}

Your adventure will take the form of two main types of actions:

\begin{enumerate}
\item Moving around the game world on the Map Screen.
\item Fighting off monsters on the Combat Screen.
\end{enumerate}

\section{The Map Screen}

The Map Screen is where you navigate the world of Aclypt. You'll
encounter different types of terrain in different provinces of the
world, but you'll get around in the same way.

To walk around the world, press the joystick controller in the
direction you'd like to travel.

As you travel, you may encounter other characters, special items, and
monsters. To interact with them, simply walk up to them until you
touch.

Monsters can be very sneaky. You may not always see monsters
approaching.

Certain obstacles in the game world may prevent you from going past
them if you do not have the proper item.


\section{The Combat Screen}

Once a peaceful kingdom, some parts of Aclypt are now riddled with
dangerous monsters.

You can approach a monster that you see on the Map Screen and start a
battle with them by running into the monster. Even though you might
only see one monster on the Map Screen, there may be more than one
traveling together. You can avoid these battles by not touching the
monster on the Map Screen.

Sometimes, though, monsters may sneak up on you. You can encounter
monsters in dangerous territories without ever having seen them. These
battles cannot be avoided.

In a normal battle, you'll see one or more monsters at the top of your
screen, along with the name of the monster type.

\subsection{On the Monsters' Turn}

When it's the monsters' turn to attack, you'll see only the monsters
on your screen.

The monsters may attack you. If they succeed, you'll lose an amount of
energy. Improving your shield will protect you from more attacks, and
reduce the amount of damage the monsters do.

If you run out of energy, your game is over. Don't worry, though ---
you can still resume your adventure from the Title Screen.

\subsection{On Your Turn}

At the bottom of the Battle Screen are four icons, which you can use
to pick your strategy.

Press up on the joystick controller to use your bow \& arrows to shoot
at the monsters. Press left to use your sword. Press right to use a
magic spell (if you know any). Pull down on the joystick to try to run
away.

When you've chosen your strategy, press the Fire button to proceed.

If you're using your sword, or bow \& arrows, you'll have to choose
which monster you'll attack. Press the joystick to select a monster,
then press the Fire button.

If you're using a magic spell, you'll see your magic scroll. Press up
and down on the joystick controller to choose the name of the magic
spell you want to use, then press the Fire button to select it.

If you hurt a monster, you'll see the amount of damage that you did on
the   screen  briefly.   Once   a  monster   is   defeated,  it   will
disappear. When all of the  monsters have been defeated, you'll return
to the Map Screen.

\section{Scoring}

When you  defeat a monster, you'll  earn points. The number  of points
you earn will increase as you defeat more difficult monsters.

Your score begins at 000000 when you start your adventure.

As your score  increases, lower level monsters will learn  to fear you
and flee. You'll have less encounters with these monsters.

\chapter{Game Over}

If you fail in your mission, your game is over. However, you have
another chance to continue.

When you continue, it'll be just as if you'd never failed in the first
place. However, you'll start over from a safe place. If you fail early
in the game, you'll return to Hapri to continue your adventure. Later,
you'll find that you can be blessed by certain characters in the
game. You'll return to the last place you were Blessed if you fail.

Just choose your game slot from the Title Screen to resume your
adventure.

\chapter{Troubleshooting}

\section{Sad Face Screen}

If you see the Sad Face screen, the game is trying to tell you that
there is a problem.

From here, you can press the Game Reset switch to return to the Title
Screen. 

\subsection{Red Sad Face Screen}

The red sad  face screen means that your AtariVox  device was not found. 
Turn off your Atari and connect an AtariVox to the right controller port. 
You may plug in speakers or headphones to your AtariVox so that you can hear
the game voices.

\subsection{White Sad Face Screen}

The white sad  face screen means that the game  has encountered an error and
cannot continue.  Please contact
\href{mailto:support@star-hope.org}{support@star-hope.org} for additional
assistance.

\subsection{Bar Code Information}

The bottom half of the screen will show some bar coded information
that describes the state of your Atari when the game stopped
working. If you can take a clear photo of this bar code, it can
sometimes help the programmers to diagnose the problem.

\section{TV goes blank when saving}

Make sure your AtariVox is connected. If your AtariVox is not
connected when the game tries to save, you may see the TV picture
remain blank while the game tries to record your progress. Just plug
in the AtariVox to the right controller port and your game should be
saved.

The game saves at various times while you play, as well as when you
press the Game Reset switch, so it's best to leave the AtariVox
connected all the time. (Besides, you won't be able to hear what
anyone says if you unplug it!)

\section{No voices}

Make sure that you have speakers plugged in to your AtariVox
device. When the game first starts the title screen, after a brief
pause, you'll hear the AtariVox announce the name of the game. If you
don't, make sure that the AtariVox is connected and the speakers are
connected, powered on, and turned up.

\chapter{Technical Notes}

The following notes are of interest to hackers only. You don't need to
understand anything in this section to play Grizzards.

\section{Game Record Slots}

There are 3 logical game slots that you can choose from for the
game. Each save game slot takes up 4 blocks (256 bytes) of storage
space on your AtariVox memory card. The following blocks are used:

\begin{enumeration}
\item Slot 1 uses memory blocks $58-$5b (addresses $1600-$16ff)
\item Slot 2 uses memory blocks $5c-$5f (addresses $1700-$17ff)
\item Slot 3 uses memory blocks $60-$63 (addresses $1800-$18ff)
\end{enumeration}

TODO --- The memory footprint might be reduced in future.

\subsection{Registration}

TODO --- register these allocations with AtariAge (
https://atariage.com/atarivox/atarivox\_mem\_list.html )

Then the following will become true:

The addresses used by the Save Game Slots are registered with
AtariAge, so they won't conflict with any other games you might play.

\subsection{Portability}

The save game records are in the same format for all \emph{Grizzards} game
cartridges, regardless of the region for which they were saved.  This means
that you can save your progress on an NTSC type system and then continue
playing on a PAL or SECAM system, or vice-versa.

\subsection{Volatility}

The save game format may change during development. Since this is an
``alpha'' quality game, it's possible that the final release may not
be able to read your saved game record.

When the final game is released, you may need to start over by erasing
your game record. See Starting Your Adventure Over on page
\ref{Starting Your Adventure Over} for instructions on erasing your
game record.

\end{document}

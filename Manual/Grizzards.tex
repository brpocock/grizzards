%
% Grizzards printable manual
%
% This needs a lot of work yet.
%
% If you're a human, skip to "\mainmatter" to skip over the preamble.
%
\documentclass[12pt,twoside,openright,book]{memoir}

% \setstocksize{9in}{6in}
% \settrimmedsize{\stockheight}{\dimexpr 6in-15mm}{*}
% \settypeblocksize{516.9312pt}{243.6962pt}{*}

% \usepackage{geometry}
% \geometry{height=6in,width=4in}
% \setstocksize{6in}{4in}
\usepackage[utf8]{inputenc}
\usepackage{babel}
%\usepackage{tgtermes}
\usepackage{hyperref}
\usepackage[protrusion=true,expansion=true]{microtype}
\fontfamily{pnc}
\chapterstyle{bianchi}

\checkandfixthelayout

\title{Grizzards Player's Guide}
\author{Bruce-Robert Pocock}

%% TV Standard

\ifdefined\TVNTSC
  \newcommand\TV{NTSC}
  \newcommand\REGION{US, Canada, Mexico, Brazil, and Japan}
\else
  \ifdefined\TVPAL
    \newcommand\TV{PAL}
    \newcommand\REGION{UK and Europe (except France)}
  \else
    \ifdefined\TVSECAM
      \newcommand\TV{SECAM}
      \newcommand\REGION{France, Russia, Africa}
    \else
      \error{Must define a TV standard}
    \fi
  \fi
\fi

%% BEGIN TITLE

\makeatletter
\def\maketitle{%
  \null
  \thispagestyle{empty}%
  \vfill
  \begin{center}\leavevmode
    \normalfont
    {\LARGE\raggedleft \@author\par}%
    \hrulefill\par
    {\huge\raggedright \@title\par}%
    \vskip 1cm
  \end{center}%
  \vfill
  \null
  \cleardoublepage
  }
\makeatother

%%% BEGIN DOCUMENT

\begin{document}

%%%\let\cleardoublepage\clearpage


\maketitle

\frontmatter

\null\vfill

\begin{flushleft}
This is the \textit{Grizzards} Player's Manual

Copyright \copyright{} 2021, Bruce-Robert Pocock

\bigskip

This  version is  for systems  in \REGION{}  using the  \TV{} television
standard. For Atari  Video Computer System CX-2600  (or Sears Tele-Games
Video  Arcade or  Atari 7800  ProSystem) with  AtariVox (or  MemCard, or
SaveKey) device.

\bigskip

This videogame software was not created, published, or licensed by Atari
or its successors.

\bigskip

\thedate

\bigskip

This  manual  is  Copyright   \copyright{}  2021,  Bruce-Robert  Pocock.
All rights reserved.

\bigskip

This is a DRAFT document.

\ifdef\DEMO
This manual describes  a DEMO version of the game.  The full version may
be different.
\fi

\end{flushleft}
\let\cleardoublepage\clearpage

\mainmatter

\chapter{Introduction}\label{Introduction}

The  land of  Syrex  is  a dangerous  place.  Fierce  monsters roam  the
countryside. But luckily for you, you're a Grizzard handler!

Train your Grizzard to  use a variety of moves to  take on the monsters.
Discover new kinds  of Grizzards with new capabilities.  Can you conquer
all the monsters of Syrex?

In the \textit{Grizzards}  videogame, you'll roam the land  looking for monsters.
Monsters may  surprise you as  you travel, or  you may see  them coming.
When faced  with terrifying beasts,  you'll direct your Grizzard  to use
its moves to defend you and attack the monsters.

\cleardoublepage

\tableofcontents

\chapter{Setting Up}\label{Setting Up}

To play \textit{Grizzards}, you will need:

\begin{itemize}
\item An  Atari Video  Computer System  CX-2600, Sears  Tele-Games Video
  Arcade, Atari 2600jr game system, or Atari 7800 ProSystem
\item A TV or video display
\item A joystick controller
\item  An AtariVox  device with  (optional) speakers  or headphones,  or
  a MemCard or SaveKey device.
\item The \textit{Grizzards} game cartridge
\end{itemize}

Set  up your  Video  Computer  System with  your  TV  or video  display.
Connect a  joystick controller to  the \emph{left} controller  port, and
the  AtariVox  (or  MemCard  or  SaveKey)  device  to  the  \emph{right}
controller  port.  Make  certain  that   the  Power  switch  is  in  the
Off position.

Finally, insert the \textit{Grizzards} game  cartridge (with the label facing up)
into the cartridge slot, and turn the Power On.

\section{Controller Ports}

Your joystick  connects to the  left controller port, and  your AtariVox
(or MemCard or SaveKey) device connects to the right controller port.

If your  AtariVox (or MemCard or  SaveKey) is missing, you'll  see a red
Sad Face screen.

This game  saves your progress to  the AtariVox (or MemCard  or SaveKey)
while you play. Always return to the Title Screen by pressing Game Reset
before turning Off the Power to be sure your game record is not damaged.

\section{Console Controls}

\ifdefined\TVSECAM
\else

\subsection{Color/B\&W Switch (Pause)}

The  Color/B\&W switch  on the  Atari Video  Computer System  CX-2600 or
Sears Tele-Games  Video Arcade (or  the Pause  button on the  Atari 7800
ProSystem) can  be used to pause  your game. Push the  Color/B\&W switch
into the Color position to play, or the B\&W position to pause the game.
On the Atari  7800, press the Pause  button once to pause  the game, and
again to resume playing.

\fi

\subsection{Game Select}

When viewing  the Title Screen,  you can use  the Game Select  switch to
choose  the  Slot  you  wish  to   use.  See  Getting  Started  on  page
\ref{Getting Started} for details.

While you are  playing the game, you  can use the Game  Select switch to
review   your  Grizzard's   statistics   from  the   Combat  screen   or
a Grizzard Depot.

On the Map screen, the Game Select switch has no effect.

\subsection{Game Reset}

When viewing  the Select  Slot screen,  press the  Game Reset  switch to
begin playing the game.

While you are  playing the game, press the Game  Reset switch to abandon
your progress and return to the Title Screen. You will lose any progress
since the last time you visited a Grizzard Depot.

\subsection{Difficulty Switches}

In  order  to erase  your  game  progress information,  both  Difficulty
Switches must be moved to the A (Expert) position.

To protect  against accidentally erasing  your game record, you  can set
either one of the Difficulty Switches to the B (Novice) position.

\ifdefined\TVSECAM

While you are  playing, the Left Difficulty Switch can  be used to Pause
the  game.  To   play,  slide  the  Left  Difficulty   Switch  into  the
B (Novice) position. To pause the game, slide the Left Difficulty Switch
into the A (Expert) position.

\else

While  you  are playing,  the  Difficulty  Switches  have no  effect  on
game play.

\fi

\section{AtariVox, SaveKey, or MemCard}

You can  use an AtariVox,  SaveKey, or MemCard  device. One of  these is
required and  must be  plugged in to  the \emph{right}  controller port.
Only plug in the memory device while the power is \emph{off}.

Each  save game  slot occupies  four blocks  of storage  on your  memory
device; all three  possible slots therefore occupy a total  of 12 blocks
of storage.

\chapter{How To Begin}\label{How To Begin}

Once your  Atari system is set  up and everything is  connected, turn on
the  Power switch.  You'll see  a  brief introduction  and title  screen
appear.  If  you   have  an  AtariVox  device,  you'll   also  hear  the
title spoken.

Press the  Game Select  switch to  choose a memory  slot for  your game.
There are  3 memory slots  possible. Press  Game Select again  to rotate
through them. If someone has already begun to play \textit{Grizzards} in
a certain slot, your screen will  show ``\texttt{RESUME}.'' If a slot is
empty, you'll see ``\texttt{BEGIN}'' instead.

When you have selected the slot you want to begin (or resume), press the
Game Reset switch to start.

\section{Choosing a Starter Grizzard}

When you begin  the game, you can  roam the game world.  You'll see that
there are four directions in which you can set out. In each direction is
a Grizzard  companion that you  can choose to  join your party.  You can
only have one Grizzard companion at a time, so choose carefully.

\chapter{How To Play}

\section{General Controls}

You'll use the joystick controller in  the left controller port for most
of your  game play needs.  Depending on where you  are in the  game, the
joystick will be used in different ways, which will be explained next.

The  Game  Select  switch  is  used from  the  title  screen  to  choose
a slot\footnote{Technical Note: The Slot  number chosen here is relative
  to  the three  save game  slots  used by  the \textit{Grizzards}  game
  program.  Each save  game  slot  actually occupies  4  blocks on  your
  AtariVox, MemCard or  SaveKey device.} in which to  save your progress
on your AtariVox (or MemCard or SaveKey). When you have selected a slot,
the Game Reset switch will start the game.

While you're playing the game, the Game Select switch can be used (from the
Combat screen or at a Grizzard Depot) to see the statistics of your current
Grizzard companion, and the Game Reset switch can be used to leave the game
and return to the title screen, saving your current progress to your
AtariVox (or MemCard or SaveKey).

\ifdefined\TVSECAM

The Left Difficulty Switch  can be used to pause game  play. When in the
``A'' (Advanced  or Expert) position, you  will not be able  to continue
playing  until  you  toggle  the   switch  to  the  ``B''  (Beginner  or
Novice) position.

You cannot delete a game in progress unless both Difficulty Switches are
in the ``A'' (Advanced or Expert) position.

\else

The Color/B\&W  switch (on an Atari  7800: Pause button) can  be used to
pause game play.  When paused, you will not be  able to continue playing
until you toggle the switch to the Color position.

This game does not make use  of the Difficulty Switches, except that you
cannot  delete a  game in  progress unless  they both  are in  the ``A''
(Advanced or Expert) position. To  protect your game from being deleted,
set either  one of  the Difficulty  Switches to  the ``B''  (Beginner or
Novice) position.

\fi

\section{Roaming The World}

\ifdef\TVNTSC
\else
The World Map screen shows your  current score (initially 000000) at the
top of the screen.
\fi
In the map display, you'll see the current area in which you
are traveling. Guide yourself using the joystick controller.

*** image

Throughout  the  world  you  may  encounter  monsters.  Sometimes  these
monsters may  sneak up on  you and attack!  Other times you'll  see them
waiting for you  and can avoid them  --- or walk up to  them when you're
ready to face them.

\ifdef\DEMO
In this demo,  you will have one Grizzard companion,  Aquax. In the full
game, you  may encounter other Grizzards  that you can convince  to join
your party.

\else

You may also  encounter Grizzards in the world that  you can convince to
join your party. When you first begin, you'll only be able to choose one
Grizzard. Later, you'll have the option to swap Grizzards in your party,
with up  to 30  total Grizzards  saved in your  AtariVox (or  MemCard or
SaveKey) device. Each Grizzard may have its own moves and skill scores.

\fi

To  review the  statistics  of your  current  Grizzard companion,  press
Game Select.

\section{Grizzard Depots}

To \ifdef\DEMO\else  swap which  Grizzard you're using,  or to  \fi heal
your current Grizzard partner, you'll need to find a Grizzard Depot like
this: ***

Touch the Grizzard Depot on the map screen to enter it.

Your  progress will  \emph{immediately} be  saved to  your AtariVox  (or
MemCard or SaveKey) device.

At  a  Grizzard  Depot  your  Grizzard partner  will  be  fully  healed.
\ifdef\DEMO\else You  can then  use the joystick  to choose  a different
Grizzard to travel with you. \fi

The Grizzard Depot main screen looks like this: ***

Here, you'll see:

\begin{description}
  
\item[DEPOT] --- indicating that this is, in fact, a Grizzard Depot
\item[Your Grizzard's name] 
\item[Your Grizzard's Portrait] 
\item[PLAYED 0002 HOURS] --- indicating  the number of hours that you've
  been playing  the game. This counts  all the time that  you've played,
  since you  first began this  adventure.

\end{description}

To review your Grizzard's statistics, press the Game Select switch.

\ifdef\DEMO\else
If you  have more than one  Grizzard in your collection,  you can choose
which  Grizzard will  be  your  companion. Press  up  (forward) or  down
(backward)  on  the  joystick  controller  to  cycle  through  available
Grizzards.
\fi

When you're ready to return to your adventure, press the Fire button.


\section{Battling Monsters}

Monsters plague the world of Syrex. If you're caught by monsters without
a Grizzard partner, they're sure to eat you alive! Luckily your Grizzard
partner  will  defend  you  from  them,  and  monsters  will  attack  it
before you.

When you encounter monsters, you'll  see the Combat display, which looks
like this: ***

The name of  the monsters that you're facing is  displayed. Monsters may
travel in groups, so you may see more than one monster facing you.

You'll also  see the name  of your  Grizzard companion and  its portrait
below. The  long bar represents  the health of  your Grizzard. If  it is
reduced to zero, your adventure will be over.

Using the joystick controller, you can  choose from among the moves that
your Grizzard knows how to perform. Press  up and down to select a move.
If your Grizzard knows  how to perform a move, it  will appear in color.
If your Grizzard does not yet know how to perform a move, it will appear
in black.

Most  Moves will  target one  monster  that you're  facing. (Some  Moves
instead target  all monsters, or yourself.)  Press left or right  on the
joystick   controller   to   select    a   target   if   you're   facing
multiple monsters.

When you  see the  selection you  want, press the  Fire button.  If your
Grizzard does  not know how  to perform  the selected move,  you'll hear
a special sound effect and nothing will happen.

To review the statistics of your Grizzard, press Game Select.

You will  see a  special display  when your Grizzard,  or a  monster, is
executing a move: ***

The creature using that move's name  will appear first, then the name of
the move,  then the  creature being  targeted. If  you have  speakers or
headphones on your AtariVox, you'll also hear the move announced.

It's possible for a move to miss its target. If that happens, you'll see
MISSED appear briefly (and hear it announced on your AtariVox).

After a move  has been executed, the creature targeted  by that move may
be injured (lose hit points) or  have its statistics changed. Changes to
statistics  are temporary  and last  only  the duration  of one  battle.
After the battle, your Grizzard's statistics will return to normal.

If  you Grizzard  loses hit  points, the  bar below  your Grizzard  will
reduce in  length. When your Grizzard  is nearly out of  hit points, the
bar will change to a red color to draw attention to that fact.

If your  Grizzard is defeated, the  monsters will surely eat  you alive.
Your adventure  will end there, and  you'll return to the  title screen.
Don't worry, though;  you can continue from the last  Grizzard Depot you
visited by resuming  your game's saved progress. Just  press Game Select
to choose the same game slot and Game Reset to resume and try again.

If you  defeat all of  the monsters, victory  is yours! Your  score will
increase, and you'll return to the World Map screen victorious.

\subsection{Grizzard Learning}

Your  Grizzard companion  may  learn from  opposing  Monsters. This  can
result  in your  Grizzard  increasing  its Attack  or  Defend score,  or
learning a Move that a monster has just performed.

Your Grizzard  can only  learn certain moves.  Moves that  your Grizzard
might be able to  perform, but does not yet know how  to, will appear in
black on the Combat display.

\section{Grizzard Depots}\label{Grizzard Depots}

A  Grizzard  Depot  is  a  place where  you  can  heal  your  Grizzard's
wounds\ifdef\DEMO\else  and swap  between  Grizzard  characters in  your
party to choose your current companion\fi.

When you encounter a Grizzard Depot, you'll see this screen: ***

Your Grizzard will  immediately be healed completely; you  don't need to
do anything else if that's what you meant to do.

The name  of your  current Grizzard companion  appears here,  along with
their  portrait,  and  the  total  number of  hours  which  you've  been
playing \textit{Grizzards}.

\ifdef\DEMO\else
To swap between Grizzards in your party, press the joystick up and down.
If you have additional Grizzards in your party, they'll switch places to
became  your  companion  immediately.  There are  30  distinct  Grizzard
companions in the game. Can you discover all of them?
\fi

You  can press  Game  Select switch  to review  the  statistics of  your
selected companion.

To leave the Grizzard Station, press the Fire button.


\section{Winning the Campaign}\label{Winning the Campaign}

\ifdef\DEMO

It is not possible to ``win'' the demo version of this game. You'll have
to wait for the full game to be released!

\else

If  you  defeat  all of  the  main  monsters  in  the game,  you'll  see
a fireworks  show to let  you know  that you've completed  the campaign.
Your ``final'' score will appear above the fireworks display.

After this  reward, you'll return  to the  Title Screen. You  can resume
play from  your save  game slot  if you'd like,  and continue  to battle
sneaky monsters that attack you in some areas.

\fi

\chapter{Getting Started}\label{Getting Started}

When you first turn on \textit{Grizzards}, you'll see the Title Screen.
To begin playing, press the Game Select switch.

In \textit{Grizzards}, your game  progress is recorded on your AtariVox
device. You have a choice of 3 ``slots'' so that three different players
can use the same AtariVox device to record their progress.

Press  Game  Select  to  choose  one of  the  three  slots,  then  press
Game Reset.

If  you've started  an adventure  in a  slot, you'll  see that  the word
``\texttt{RESUME}'' appears on your screen. If that slot is empty, you'll instead
see the word ``\texttt{BEGIN}.''

\section{Resuming Your Progress}

Your progress  can be saved  by visiting a  Grizzard Depot in  the game.
When you resume playing later, you'll  return to the last Grizzard Depot
that you  visited. If you haven't  yet visited a Grizzard  Depot, you'll
return to the original starting point of the game.

\section{Starting Your Adventure Over}\label{Starting Your Adventure Over}

When you choose a  slot with no game record in  it already, you'll begin
a new adventure.

If you want to delete your adventure  and start again, it's a little bit
tricky. This is to make sure you don't accidentally lose your progress!

From the Title Screen, press Game  Select. Then, press Game Select until
the slot you want to erase is shown.

Here's the tricky part. You'll need to:

\begin{itemize}
\item Make sure that both of the Difficulty Switches on your console
  are set to the ``A'' (Advanced or Expert) position.
\item With your joystick controller, pull down (toward you) on the
  joystick and hold down the Fire button.
\end{itemize}

The screen  will change from  saying ``\texttt{SELECT SLOT}''  to saying
``\texttt{ERASE  SLOT}.''   The  text   will  also   be  red   to  catch
your attention.

If  you're  \emph{sure}   you  want  to  erase  your   game  data,  then
\emph{without}  letting go  of the  Fire  button, move  the joystick  up
(toward  the TV).  You'll see  that the  slot changes  from ``\texttt{IN
  USE}'' to ``\texttt{VACANT}'' \emph{immediately}.

\emph{Once your  game record has been  erased, there's no way  to get it
  back, so think carefully before you erase it.}

\subsection{Protecting Your Game Record}

If  either of  your Difficulty  Switches is  in the  ``B'' (Beginner  or
Novice) position, then you can't erase  a game slot. \emph{Tip: When you
  connect your AtariVox, check the position of those switches.}

\ifdef\TVSECAM
Remember,  the  Left  Difficult  Switch  is  used  to  pause  the  game.
After deleting your save game slot, return the Left Difficulty Switch to
the ``B'' position.
\fi

\subsection{Missing AtariVox}

If your  AtariVox (or MemCard or  SaveKey) is missing, you'll  see a red
Sad Face screen.

Always  turn Off  the Power  switch  before attaching  or removing  your
AtariVox (or MemCard or SaveKey).

This game  saves your progress to  the AtariVox (or MemCard  or SaveKey)
while you play. Always return to the Title Screen by pressing Game Reset
before turning Off the Power to be sure your game record is not damaged.

\chapter{Interacting With the Game}

Your adventure will take the form of two main types of actions:

\begin{enumerate}
\item Moving around the game world on the Map Screen.
\item Fighting off monsters on the Combat Screen.
\end{enumerate}

\section{The Map Screen}

The Map Screen is where you navigate the world.  You'll encounter different
types of terrain in different provinces of the world, but you'll get around
in the same way.

To walk around the world, press the joystick controller in the direction
you'd like to travel.

As you travel,  you may encounter monsters,  Grizzards, Grizzard Depots,
or doors. To interact with them, simply walk into them.

A monster, or group of monsters, look like this: ***

A wild Grizzard looks like this: ***

A Grizzard Depot looks like this: ***

A door looks like this: ***

Monsters  can  be   very  sneaky.  You  may  not   always  see  monsters
approaching. As you travel, sneaky monsters may attack and draw you into
combat unexpectedly.

\section{The Combat Screen}

You can approach a monster that you see on the Map Screen and start a
battle with them by running into the monster. Even though you might
only see one monster on the Map Screen, there may be more than one
traveling together. You can avoid these battles by not touching the
monster on the Map Screen.

Sometimes,  though, monsters  may sneak  up  on you.  You can  encounter
monsters  in  dangerous  territories  without  ever  having  seen  them.
These battles cannot be avoided.

In a battle, you'll see one or  more monsters at the top of your screen,
along with the name of the monster type. Below this, you'll see the name
of your Grizzard  companion and its health bar. The  longer this bar is,
the more health your Grizzard companion has.

\subsection{On the Monsters' Turn}

When it's the monsters' turn to attack, you won't see any move selection
menu at the bottom of the screen.

The monsters may  attack your Grizzard. If they succeed,  it will reduce
the health of your Grizzard.

If  your  Grizzard  is  defeated,  the monsters  will  surely  eat  you.
Your  adventure  will  end,  and  you'll return  to  the  Title  screen.
Don't worry,  though; your  progress will  be saved  still, and  you can
resume at the last Grizzard Depot that you had visited.

\subsection{On Your Turn}

When it's your turn, you'll see the name of a Move that your Grizzard may be
able to perform at the bottom of the screen. If your Grizzard knows how to
perform this Move, it will appear in color. If your Grizzard does not know
how to perform this Move, it will appear in black.

Select a Move by pressing the joystick up and down (forward and back).

Most Moves will target one of the monsters attacking you. When there are
multiple monsters, you must choose which one to counter-attack. Select the
target by pressing the joystick left and right.

When you've selected your Move and target, press the Fire button.

\subsection{Statistics}

You can also press the Select switch to review your Grizzard's statistics.
Each Grizzard companion has a few details:

\begin{description}
  
\item[\texttt{ATK}] is the Grizzard's  \emph{attack} rating. This is the
  amount  of damage  that  your  Grizzard can  do  when it  successfully
  attacks a monster. Some Moves cause more damage than others, though.
  
\item[\texttt{DEF}] is the Grizzard's  \emph{defend} rating. This is how
  likely your Grizzard is to avoid being hurt by a monster's Move.

\item[\texttt{HP}]  is  the  Grizzard's  \emph{hit  points}  or  health.
  When  monsters hit  your Grizzard,  this  value will  decrease. If  it
  reaches zero, your game is over.

\item[\texttt{MAX}]  is   the  Grizzard's  \emph{maximum   hit  points}.
  Your Grizzard can gain more hit points up to this amount.
  
\end{itemize}

Each of these statistics can be raised up to a maximum of 99 points.

Your  Grizzard may  also  be  subject to  one  or  more Status  Effects.
These  will  alter your  Grizzard's  status  temporarily (only  for  the
duration of  one battle). These effects  are displayed at the  bottom of
the screen. If  there are more than  one Status Effect in  effect at the
same  time,  only one  will  be  displayed; wait  a  moment  to see  the
others appear.

\section{Scoring}

When you  defeat a monster, you'll  earn points. The number  of points
you earn will increase as you defeat more difficult monsters.

Your score begins at 000000 when you start your adventure.

\chapter{Game Over}

If  you fail  in your  mission,  your game  is over.  However, you  have
another chance to continue.

When you continue, it'll  be just as if you'd never  failed in the first
place. However, you'll start over from  the last Grizzard Depot that you
had visited. If you fail before visiting a Grizzard Depot, you'll return
to the original starting point.

Just  choose   your  game   slot  from  the   Title  Screen   to  resume
your adventure.

\chapter{Grizzards}

There are  30 Grizzards in  the game world,  each with their  own unique
starting attributes and sets of Moves.

\ifdef\DEMO

In this  demo, you  can play  with only Aquax.  Other Grizzards  will be
available in the full game.

\fi

Each Grizzard is able  to learn up to 8 different  moves, in addition to
the universal move RUN AWAY. It's up to you to discover which Moves each
Grizzard  is able  to  learn. See  Moves on  page  \ref{Moves} for  more
information about the Moves that Grizzards and monsters can use.

Here  is a  list of  some of  the Grizzards  that you  can encounter  in
the game.

\subsubsection{Dirtex}

Dirtex is  an Earth-type Grizzard.  It can  learn moves that  reduce the
effectiveness of enemy monsters, as well as doing damage directly.

\subsubsection{Aquax}

Aquax is a Water-type Grizzard. It can learn ***

\subsubsection{***}

\chapter{Moves}\label{Moves}

There are  many different Moves that  Grizzards and monsters can  use in
the  game.  Some Moves  are  used  to  reduce  the enemy's  hit  points.
If a Grizzard (or monster) runs out of hit points, they are defeated.

If all  monsters in a  group are defeated, they  are gone from  the game
for good. If your Grizzard is defeated, your game is over.

\section{Run Away}

This is a special Move that you can use to escape from combat and return
to the map screen.

Your  Grizzard will  not be  healed if  you run  away (unless  you visit
a  Grizzard Depot);  however, the  monsters  that you  were facing  will
be healed.

\section{Move Effects}

Each  Move  can have  a  variety  of effects  on  your  Grizzard or  the
enemy monsters. In addition to  causing damage (reducing the enemy's hit
points), a Move  can also be used to heal  from damage (increasing one's
own hit points).

A Move  can also  reduce an  enemy's statistics,  or increase  one's own
statistics. These are  called Status Effects, and they  persist for some
time after  they are applied. If  any Status Effects are  affecting your
Grizzard, they'll appear at the bottom of the statistics screen.

There  are  other   status  effects  that  you  can   also  discover  as
you explore.

\section{Learning Moves}

Grizzards may learn a move when they see an enemy monster using it.

\section{Some Moves}

Here  are  descriptions  of  some  of  the  Moves  that  your  Grizzards
might learn. There are more than 50 Moves possible, many of which can be
learned by your Grizzards.

\subsection{Run Away}

This Move  lets you  escape from  a battle. While  your Grizzard  is not
healed, the monsters will be, so be careful in deciding when to flee.

\subsection{Kick Dirt}

In this  move, the attacker  kicks dirt at  their enemy. This  can cause
minor damage to the enemy.

\subsection{***}

\chapter{Troubleshooting}

\section{Sad Face Screen}

If you  see the Sad  Face screen,  the game is  trying to tell  you that
there is a problem.

From  here,  you can  press  the  Game Reset  switch  to  return to  the
Title Screen.

\subsection{Red Sad Face Screen}

The red sad  face screen means that your AtariVox  device was not found. 
Turn off your Atari and connect an AtariVox to the right controller port. 
You may plug in speakers or headphones to your AtariVox so that you can hear
the game voices.

\subsection{White Sad Face Screen}

The white sad  face screen means that the game  has encountered an error
and cannot continue. Please contact
\href{mailto:support@star-hope.org}{support@star-hope.org} for 
additional assistance.

You should not be able to reach this screen.

\subsection{Bar Code Information}

The bottom half of the screen will show some bar coded information
that describes the state of your Atari when the game stopped
working. If you can take a clear photo of this bar code, it can
sometimes help the  programmers to diagnose the problem,  but no worries
if you can't.

\section{TV goes blank when saving}

Make sure  your AtariVox is connected.  If your AtariVox (or  MemCard or
SaveKey) is not connected  when the game tries to save,  you may see the
TV picture remain blank while the game tries to record your progress.

\section{No voices}

Make sure  that you have  speakers plugged  in to your  AtariVox device.
When the game first starts the title screen, after a brief pause, you'll
hear the AtariVox announce the name of the game. If you don't, make sure
that  the AtariVox  is connected  and the  speakers (or  headphones) are
connected, powered on, and turned up.

Naturally,  there  are  no  voices   when  playing  with  a  MemCard  or
SaveKey device. 

\chapter{Technical Notes}

The following notes are of interest to hackers only. You don't need to
understand anything in this section to play \textit{Grizzards}.

\section{Development Tools}

The \textit{Grizzards}  source code and development  tools are available
from
\href{https://Star-Hope.org/games/Grizzards/}{https://Star-Hope.org/games/Grizzards/},
the \textit{Grizzards} web site.

Development of \textit{Grizzards} made use of:

\begin{itemize}
\item Fedora Linux\registered{} operating system
\item 64tass Turbo Assembler Macro
\item Steel Bank Common Lisp
\item GNU Make
\item The GNU Image Manipulation Program (Gimp)
\item GNU Emacs
\item The Stella emulator
\item The \TeX{} typesetting system (for this manual)
\end{itemize}

\section{Game Record Slots}

There are 3  logical game slots that  you can choose from  for the game.
Each save game  slot takes up 4  blocks (256 bytes) of  storage space on
your AtariVox memory card,  for a total of 12 blocks  for all three save
game slots. The following blocks are used:

\begin{enumeration}
\item Slot 1 uses memory blocks \$5c-\$5f (addresses \$1700-\$17ff)
\item Slot 2 uses memory blocks \$60-\$63 (addresses \$1800-\$18ff)
\item Slot 3 uses memory blocks \$64-\$67 (addresses \$1900-\$19ff)
\end{enumeration}

\subsection{Registration}

\ifdef\FIXMERegisterGameWithAtariAge

The addresses used by the Save Game Slots are registered with
AtariAge, so they won't conflict with any other games you might play.

\else

The save game blocks used are \emph{not yet} registered with AtariAge.

\fi

\subsection{Portability}

The save game records are in the same format for all \textit{Grizzards} game
cartridges, regardless of the region for which they were saved.  This means
that you can save your progress on an NTSC type system and then continue
playing on a PAL or SECAM system, or vice-versa.

\subsection{Volatility}

The save game format may change during development. Since this is an
``alpha'' quality game, it's possible that the final release may not
be able to read your saved game record.

When the final game is released, you may need to start over by erasing
your game record. See Starting Your Adventure Over on page
\ref{Starting Your Adventure Over} for instructions on erasing your
game record.

\chapter{Credits}

The  \textit{Grizzards} videogame  software,  including its  audiovisual
components   and  this   manual,   are   copyright  \copyright{}   2021,
Bruce-Robert  Pocock.   All  Rights  are  Reserved   except  as  granted
under license.

\begin{itemize}
  
\item Bruce-Robert Pocock \hdots Programming, Manual, and Artwork
\item  Includes VCS  header  file by  Matthew  Dillon, Olaf  ``Rhialto''
  Seibert, Andrew David, and Peter H. Froehlich
\item Binary-to-decimal  translation based  upon code by  Andrew Jacobs,
  based upon code by Garth Wilson
\item Special thanks to everyone  in the Stella and AtariAge communities
  for making this game possible.
\end{itemize}

The \textit{Grizzards} videogame software has not yet been published.

\subsection{License}

You are hereby granted permission  to make use of the \textit{Grizzards}
videogame software for \emph{non-commercial personal use}.

Redistribution not for profit is allowed, but sales of this software
requires a license.

\end{document}
